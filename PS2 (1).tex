%------------%
%  Preamble  %
%------------%

\documentclass[final,11pt]{article}
\usepackage[paperwidth=9.0in, top=1.2in, bottom=1.2in, left=1.2in, right=1.2in]{geometry}
\usepackage{amsmath}
\usepackage{color}
\usepackage{multirow}
\usepackage{setspace}
\usepackage{fancyhdr}
\usepackage{longtable}
\usepackage{array}
\usepackage{booktabs}
\usepackage{mathpazo}
\usepackage{threeparttable}
\usepackage{eurosym}
\usepackage{graphicx}
\usepackage{float}
\usepackage[colorlinks, linkcolor=blue, anchorcolor=blue, citecolor=blue]{hyperref}

\renewcommand{\headrulewidth}{0pt}
\setlength{\arraycolsep}{10pt}
\setlength\headheight{0.5cm}
\setlength\headsep{0.8cm}
\setlength\footskip{1.0cm}
\setlength{\parindent}{0em}
\pagestyle{fancy}
\chead{\textcolor[rgb]{0.5,0.5,0.5}{\sc Spring 2025: ECON 6100}}

%------------%
%  Document  %
%------------%

\begin{document}
\thispagestyle{empty}
\begin{spacing}{1.25}

\textbf{Robby Kevlishvili \hfill Problem Set 2, Due: Mar. 18, 2025}\\

\underline{All the Python codes will be added on the last page, in the Appendix.} \\

(1) Use the probability integral transformation method to simulate from the distribution
\begin{gather}
    f(x) = 
    \begin{cases}
        \frac{2}{a^2}x,  & \text{if }0\leq x\leq a \\
        0, & \text{otherwise}
    \end{cases}
\end{gather}
where $a>0$. Set a value for $a$, simulate various sample sizes, and compare results to the true distribution.

{\color{blue}
Solution:
\begin{enumerate}
    \item The function F(X) is 
    \begin{gather}
        (1/a^2) x^2 \ for \ 0 \leq  x \leq a 
    \end{gather}
    Drawing $u \sim U(0,1)$ 
    \begin{gather}
        x=a\sqrt u \sim f(x)
    \end{gather}
    Plotting the graphs:
    \begin{figure} [H]
        \centering
        \includegraphics[width=0.65\linewidth]{a=10-histogram.png}
        \caption{Distribution with a = 10, n = 10,000}
        \label{fig:enter-label}
    \end{figure}
\begin{figure} [H]
    \centering
    \includegraphics[width=0.65\linewidth]{a=10-n=100,000.png}
    \caption{Distribution with a = 10, n = 100,000}
    \label{fig:enter-label}
\end{figure}
\begin{figure} [H]
    \centering
    \includegraphics[width=0.65\linewidth]{true-dist.png}
    \caption{True Distribution of the PDF}
    \label{fig:enter-label}
\end{figure}
\end{enumerate}
}

We see that as n increases, the distribution becomes a lot more 'smooth', and more closely resembles the true distribution of the PDF.

\newpage

(2) Generate samples from the distribution
\begin{gather}
    f(x)=\frac{2}{3}e^{-2x}+2e^{-3x}
\end{gather}
using the finite mixture approach.

{\color{red}
Solution:
\begin{enumerate}
    \item The PDF of f(x) can be written as 
    \begin{gather}
        f(x) = \frac{1}{3}2e^{-2x} + \frac{2}{3}3e^{-3x}
    \end{gather}
    The CDF of an exponential distribution with rate $\lambda$ is 
    \begin{gather}
        F(x) = 1-e^{-\lambda x}
    \end{gather}
    The inverse of the CDF for an exponential distribution with rate $\lambda$ is 
    \begin{gather}
        F^{-1}(u) = -\frac{1}{\lambda} ln(1-u), with \; u \; \epsilon \; (0,1) 
    \end{gather}
    Steps: \\
    Generate a random number $u_1$ from a uniform distribution between 0 and 1 \\
    If $u_1 < \frac{1}{3}$, select the first component $(\lambda_1 = 2)$. If $u_1 > \frac{2}{3}$, select the second component $(\lambda_1 = 3)$ \\
    If the first component was selected, generate a random number $u_2$ from a uniform distribution between 0 and 1. Calculate $x = - \frac{1}{2} ln(1-u_2)$.\\
    If the second component was selected, generate a random number $u_2$ from a uniform distribution between 0 and 1. Calculate $x = - \frac{1}{3} ln(1-u_2)$.\\

    Plotting the graph:
\begin{figure} [H]
    \centering
    \includegraphics[width=0.7\linewidth]{finite-mixture-approach-graph.png}
    \label{fig:enter-label}
\end{figure}

\end{enumerate}
}


(3) Draw 500 observations from Beta$(3,3)$ using the accept-reject algorithm. Compute the mean and variance of the sample and compare them to the true values. $\to$ \\  \textbf{ Calculated values included in graph. Overall close resemblance.}\\

\begin{figure} [H]
    \centering
    \includegraphics[width=0.7\linewidth]{accept-reject-graph.png}
    \label{fig:enter-label}
\end{figure}

\newpage

\textbf{Appendix}

{\color{blue}
    \textbf{Python Code for Q1:} \\
    import numpy as np \\
    import scipy.stats as stats \\
    import matplotlib.pyplot as plt \\
    \textbf{Generate random numbers} \\
    a = 2 \\
    u = np. random . rand (10000) \\
    x = a * u **(1/2)  \\
    \textbf{Plot} \\ 
    plt . hist (x, bins =50 , density =True , color ="red ", alpha =0.5) \\
    plt.xlabel ("x") \\
    plt.ylabel (" Probability Density ") \\
    plt.title (" Histogram ") \\
    plt.show () \\
}

{\color{red}
 \textbf{Python Code for Q2:} \\
    import numpy as np \\
    import matplotlib.pyplot as plt\\

    # Parameters
    n_samples = 10000  # Number of samples \\
    weights = [1/3, 2/3]  # Mixture weights for components \\ (λ=2 and λ=3) \\
    lambdas = [2, 3]  # Rate parameters for the exponential components \\

    # Step 1: Choose components based on weights\\
    u1 = np.random.rand(n_samples)\\
    component_choice = np.where(u1 < weights[0], 0, 1)  # 0 for λ=2, 1 for λ=3 \\

    # Step 2: Sample from the selected exponential components \\
    x = np.zeros(n_samples) \\
        for i in range(len(lambdas)): \\
        mask = (component_choice == i) \\
        n_selected = np.sum(mask) \\
        # Inverse transform method: X = -ln(U)/λ \\
        x[mask] = -np.log(np.random.rand(n_selected)) / lambdas[i] \\

    # Plot histogram vs theoretical PDF \\
    plt.hist(x, bins=50, density=True, alpha=0.6, color="red", label="Samples") \\

    # Overlay theoretical PDF \\
    x_grid = np.linspace(0, 5, 500) \\
    theoretical_pdf = (weights[0] * 2 * np.exp(-2 * x_grid)) + (weights[1] * 3 * np.exp(-3 * x_grid)) \\
    plt.plot(x_grid, theoretical_pdf, "k-", linewidth=2, label="Theoretical PDF") \\

    plt.xlabel("x") \\
    plt.ylabel("Probability Density") \\
    plt.title("Finite Mixture Sampling: $f(x) = \\frac{1}{3} \cdot 2e^{-2x} + \\frac{2}{3} \cdot 3e^{-3x}$") \\
    plt.legend() \\
    plt.show() \\
}

{\color{blue}
\textbf{Python code for Q3:} \\
def target (x) : \\
return stats.beta.pdf(x, a=3 , b =3) \\
def proposal (x) : \\
return stats.uniform.pdf(x) \\
def accept _ reject (target ,proposal,c,n) : \\
sample = [] \\
while len(sample) < n: \\
x = np.random.uniform (0,1) \\
u = np.random.uniform (0,1) \\
if u <= target(x) / (proposal(x) *c) : \\
sample.append(x) \\
return np.array(sample) \\
# Sample beta \\
c = target (0.5) / proposal (0.5) \\
sample = accept_reject ( target , proposal , c, 10000) \\
}

\end{spacing}
\end{document}
